\documentclass[
	% -- opções da classe memoir --
	12pt,				% tamanho da fonte
	openright,			% capítulos começam em pág ímpar (insere página vazia caso preciso)
	oneside,			% para impressão em verso e anverso. Oposto a oneside
	a4paper,			% tamanho do papel. 
	% -- opções da classe abntex2 --
	%chapter=TITLE,		% títulos de capítulos convertidos em letras maiúsculas
	%section=TITLE,		% títulos de seções convertidos em letras maiúsculas
	%subsection=TITLE,	% títulos de subseções convertidos em letras maiúsculas
	%subsubsection=TITLE,% títulos de subsubseções convertidos em letras maiúsculas
	% -- opções do pacote babel --
	english,			% idioma adicional para hifenização
	french,				% idioma adicional para hifenização
	spanish,			% idioma adicional para hifenização
	brazil				% o último idioma é o principal do documento
	]{abntex2}

% ---
% Pacotes básicos 
% ---
\usepackage{chngcntr}
\counterwithout{footnote}{chapter}
\counterwithout{equation}{chapter}
\usepackage{lmodern}			% Usa a fonte Latin Modern			
\usepackage[T1]{fontenc}		% Selecao de codigos de fonte.
\usepackage[utf8]{inputenc}		% Codificacao do documento (conversão automática dos acentos)
\usepackage{lastpage}			% Usado pela Ficha catalográfica
\usepackage{indentfirst}		% Indenta o primeiro parágrafo de cada seção.
\usepackage{color}				% Controle das cores
\usepackage{graphicx}			% Inclusão de gráficos
\usepackage{microtype} 			% para melhorias de justificação
\usepackage{afterpage}
\usepackage{amsmath}
\usepackage{amssymb,url}
\usepackage{hhline}
\usepackage{xcolor,tikz,bm,colortbl}
\usepackage[ruled,linesnumbered,vlined,portuguese,onelanguage]{algorithm2e}
\usetikzlibrary{decorations.pathreplacing,calc}

\newcommand{\tikzmark}[1]{\tikz[overlay,remember picture] \node (#1) {};}

\renewcommand{\algorithmcfname}{ALGORITMO}

% Algumas mudanças devem ser realizadas neste arquivo, por exemplo, financiadora do projeto
\usepackage{lib/customizacoes}

% ---
		
% ---
% Pacotes adicionais, usados apenas no âmbito do Modelo Canônico do abnteX2
% ---
\usepackage{lipsum}				% para geração de dummy text
% ---

% ---
% Pacotes de citações
% ---
\usepackage[brazilian,hyperpageref]{backref}	 % Paginas com as citações na bibl
\usepackage[alf, bibjustif, abnt-etal-list=0, abnt-etal-cite=3]{abntex2cite}	% Citações padrão ABNT
% --- 
% CONFIGURAÇÕES DE PACOTES
% --- 

% ---
% Configurações do pacote backref
\renewcommand{\familydefault}{\sfdefault}
% Usado sem a opção hyperpageref de backref
\renewcommand{\backrefpagesname}{}
% Texto padrão antes do número das páginas
\renewcommand{\backref}{}
% Define os textos da citação
\renewcommand*{\backrefalt}[4]{
	\ifcase #1 %
		%
	\or
		%
	\else
		%
	\fi}%
% ---

% ---
% Informações de dados para CAPA e FOLHA DE ROSTO
% ---
\titulo{Título da Dissetação}
\autor{Nome do Autor}
\local{São José do Rio Preto}
\data{2021}
\orientador{Prof. Dr. Orientador}
\instituicao{}
\tipotrabalho{Dissertação de Mestrado Acadêmico}
% O preambulo deve conter o tipo do trabalho, o objetivo, 
% o nome da instituição e a área de concentração 
\preambulo{Dissertação apresentada como parte dos requisitos para obtenção do título de Mestre em Ciência da Computação, junto ao Programa de Pós-Graduação em Ciência da Computação, do Instituto de Biociências, Letras e Ciências Exatas da Universidade Estadual Paulista ``Júlio de Mesquita Filho", Câmpus de São José do Rio Preto.}
% ---


% ---
% Configurações de aparência do PDF final

% alterando o aspecto da cor preta
\definecolor{black}{RGB}{0,0,0}

% informações do PDF
\makeatletter
\hypersetup{
     	%pagebackref=true,
		pdftitle={\@title}, 
		pdfauthor={\@author},
    	pdfsubject={\imprimirpreambulo},
	    pdfcreator={LaTeX com abnTeX2},
		pdfkeywords={abnt}{latex}{abntex}{abntex2}{dissertação}, 
		colorlinks=true,       		% false: boxed links; true: colored links
    	linkcolor=black,          	% color of internal links
    	citecolor=black,        		% color of links to bibliography
    	filecolor=magenta,      		% color of file links
		urlcolor=black,
		bookmarksdepth=4
}
\makeatother
% --- 

% --- 
% Espaçamentos entre linhas e parágrafos 
% --- 

% O tamanho do parágrafo é dado por:
\setlength{\parindent}{1.3cm}

% Controle do espaçamento entre um parágrafo e outro:
\setlength{\parskip}{0.2cm}  % tente também \onelineskip

% ---
% compila o indice
% ---
\makeindex
% ---

% ----
% Início do documento
% ----
\begin{document}

% Seleciona o idioma do documento (conforme pacotes do babel)
%\selectlanguage{english}
\selectlanguage{brazil}

% Retira espaço extra obsoleto entre as frases.
\frenchspacing 

% ----------------------------------------------------------
% ELEMENTOS PRÉ-TEXTUAIS
% ----------------------------------------------------------
% \pretextual

% ---
% Capa
% ---
\imprimircapa
% ---

% ---
% Folha de rosto
% (o * indica que haverá a ficha bibliográfica)
% ---
\imprimirfolhaderosto*
% ---

% ---
% Inserir a ficha bibliografica
% ---

% Isto é um exemplo de Ficha Catalográfica, ou ``Dados internacionais de
% catalogação-na-publicação''. Você pode utilizar este modelo como referência. 
% Porém, provavelmente a biblioteca da sua universidade lhe fornecerá um PDF
% com a ficha catalográfica definitiva após a defesa do trabalho. Quando estiver
% com o documento, salve-o como PDF no diretório do seu projeto e substitua todo
% o conteúdo de implementação deste arquivo pelo comando abaixo:
%
% \begin{fichacatalografica}
%     \includepdf{fig_ficha_catalografica.pdf}
% \end{fichacatalografica}

\begin{fichacatalografica}
	\sffamily
	\vspace*{\fill}					% Posição vertical
	\begin{center}					% Minipage Centralizado
		\fbox{\begin{minipage}[c][8cm]{14cm}		% Largura
			\small
			\hspace{1cm} Sobrenome, Autor.
			%Sobrenome, Nome do autor
				
			\hspace{1.5cm} \imprimirtitulo~/ \imprimirautor. --
			\imprimirlocal, \imprimirdata
				
			\hspace{1.5cm} \pageref{LastPage} f. : il., tabs.\\
				
			\hspace{1.5cm} \imprimirorientadorRotulo~\imprimirorientador
				
			\hspace{1.5cm}
			Dissertação (mestrado) - Universidade Estadual Paulista ``Júlio de Mesquita Filho", Instituto de Biociências, Letras e Ciências Exatas
				
			\hspace{1.5cm}
			\\
				
			\hspace{1.5cm}
			1. Item 1.
			2. Item 2.
			3. Item 3.
			I. Sobrenome, Nome do Orientador
			II. Universidade Estadual Paulista ``Júlio de Mesquita Filho", Instituto de Biociências, Letras e Ciências Exatas.
			III. Título.
			\begin{flushright}
				CDU -- 518.72:76	
			\end{flushright}
			
			\end{minipage}}
			
		Ficha catalográfica elaborada pela biblioteca do IBILCE
		\\
		UNESP - Câmpus de São José do Rio Preto
	\end{center}
\end{fichacatalografica}
% ---

% ---
% Inserir folha de aprovação
% ---

% Isto é um exemplo de Folha de aprovação, elemento obrigatório da NBR
% 14724/2011 (seção 4.2.1.3). Você pode utilizar este modelo até a aprovação
% do trabalho. Após isso, substitua todo o conteúdo deste arquivo por uma
% imagem da página assinada pela banca com o comando abaixo:
%
% \includepdf{folhadeaprovacao_final.pdf}
%
\begin{folhadeaprovacao}
	
	\begin{center}
		{\ABNTEXchapterfont\large\imprimirautor}
		
		\vspace*{\fill}\vspace*{\fill}
		\begin{center}
			\ABNTEXchapterfont\bfseries\Large\imprimirtitulo
		\end{center}
		\vspace*{\fill}
		    
		\hspace{.45\textwidth}
		\begin{minipage}{.5\textwidth}
			\imprimirpreambulo
			\\~\\
			Financiadora: FAPESP - Proc. 2019/00000-0
		\end{minipage}%
		\vspace*{\fill}
	\end{center}
	        
	\center Comissão Examinadora
	  
	\begin{flushleft}
		\textbf{\imprimirorientador} \\ UNESP - Câmpus de Bauru \\ Orientador
		\\~\\
		\textbf{Convidado Externo} \\ Instituição
		\\~\\
		\textbf{Convidado Interno} \\ UNESP - Câmpus de Bauru
	\end{flushleft}
	      
	\begin{center}
		\vspace*{0.5cm}
		\par
		{São José do Rio Preto \\ 1 de janeiro de 2019}
		\vspace*{1cm}
	\end{center}
	  
\end{folhadeaprovacao}
% ---

% ---
% Dedicatória
% ---
\begin{dedicatoria}
	\vspace*{\fill}
	\centering
	\noindent
	\textit{ Dedicatória.} \vspace*{\fill}
\end{dedicatoria}
% ---

% ---
% Agradecimentos
% ---
\begin{agradecimentos}
	
	Agradecimentos.
	
\end{agradecimentos}
% ---

% ---
% Epígrafe
% ---
\begin{epigrafe}
	\vspace*{\fill}
	\begin{flushright}
		\textit{Epígrafe}
	\end{flushright}
\end{epigrafe}
% ---

% ---
% RESUMOS
% ---

% resumo em português
\setlength{\absparsep}{18pt} % ajusta o espaçamento dos parágrafos do resumo
\begin{resumo}
	
	Resumo.
	
	\textbf{Palavras-chave:} Palavra 1. Palavra 2.
\end{resumo}

% resumo em inglês
\begin{resumo}[Abstract]
	\begin{otherlanguage*}{english}
		 
		Abstract.
		
		\textbf{Keywords:} Keyword1. Keyword2. 
		
	\end{otherlanguage*}
\end{resumo}
% ---

% ---
% inserir lista de ilustrações
% ---
\pdfbookmark[0]{\listfigurename}{lof}
\listoffigures*
\cleardoublepage
% ---

% ---
% inserir lista de tabelas
% ---
\pdfbookmark[0]{\listtablename}{lot}
\listoftables*
\cleardoublepage
% ---

% ---
% inserir lista de abreviaturas e siglas
% ---
\begin{siglas}
	\item[S1] Sigla 1
	\item[S2] Sigla 2
	
\end{siglas}
% ---

% ---
% inserir o sumario
% ---
\pdfbookmark[0]{\contentsname}{toc}
\tableofcontents*
\cleardoublepage
% ---



% ----------------------------------------------------------
% ELEMENTOS TEXTUAIS
% ----------------------------------------------------------
\pagestyle{simple}

% ----------------------------------------------------------
% Introdução (exemplo de capítulo sem numeração, mas presente no Sumário)
% ----------------------------------------------------------

\section{Introdução}
\label{s.introduction}

\begin{frame}{Introdução}
	Lorem ipsum ...
\end{frame}
\chapter{Metodologia}
\label{c.metodologia}

\section{Bases de Dados}
\label{s.dataset}

\section{Configuração Experimental}
\label{s.experimental_setup}
\chapter{Experimentos e Resultados}
\label{c.results}
\chapter{Conclusão}
\label{c.conclusao}

\section{Direcionamentos para Trabalhos Futuros}
\label{s.future_work}

\section{Publicações Realizadas}
\label{s.publication}

% ---
% Capitulo com exemplos de comandos inseridos de arquivo externo 
% ---
% ---

% ----------------------------------------------------------
% ELEMENTOS PÓS-TEXTUAIS
% ----------------------------------------------------------
\postextual
% ----------------------------------------------------------

% ----------------------------------------------------------
% Referências bibliográficas
% ----------------------------------------------------------
\pagestyle{empty}
\bibliography{chapters/references}

% Apêndice
%

% ---
% Inicia os apêndices
% ---
%\begin{apendicesenv}
%
%% Imprime uma página indicando o início dos apêndices
%\partapendices
%
%\end{apendicesenv}

% INDICE REMISSIVO
%---------------------------------------------------------------------
\phantompart
\printindex
%---------------------------------------------------------------------

\end{document}